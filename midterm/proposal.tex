\documentclass[conference]{IEEEtran}
\makeatletter
\newcommand{\rmnum}[1]{\romannumeral #1}
\newcommand{\Rmnum}[1]{\expandafter\@slowromancap\romannumeral #1@}
\makeatother

\usepackage{epsfig}
\usepackage{amsopn}
\usepackage{subfigure}
\usepackage{cite}
\ifCLASSINFOpdf

\else
\fi
\usepackage{url}
\usepackage[cmex10]{amsmath}
\interdisplaylinepenalty = 2500

\usepackage{algorithmic}
\usepackage{algorithm}
\hyphenation{op-tical net-works semi-conduc-tor}


\begin{document}
\title{Midterm Report}
\author{\authorblockN{Quan~Wang\authorrefmark{1}, Tingwu~Wang\authorrefmark{1}, Wenxin~Wang\authorrefmark{1}, and Tianpen~Li\authorrefmark{1}}
      \small\authorblockA{\authorrefmark{1}Department of Electronic Engineering, Tsinghua University, Beijing, 100084, P. R. China\\
        E-mail: wtw12@mails.tsinghua.edu.cn, wangquan.thu@aliyun.com, stieizc.33@gmail.com, 370756595@qq.com}
	}
\maketitle
\section{Paper Reading and coding}
During the past months, we read the following papers:
\subsubsection{Benenson's Group}
Benenson's group provides the state-of-art performance of real time pedestrian detection \cite{R1,R2,R3,R4,R5}.
But it is also clear that the gpu required codes are not suitable for reproducing.
It is difficult to implement, as none of us are equipped with the required hardware.
\subsubsection{Subhransu Maji's Group}
The paper \cite{M1} discuss straightforward classification using kernelized SVMs
They evaluate the kernel for a test vector and each of the support vectors. 
But the algorithm is kind of out of date. 
And although we succeeded in rerunning it, but we choose to drop it.
\subsubsection{Ouyang's Group}
After reading \cite{W1,W2,W3,W4,W5}, we believe it is the right group to follow.
In their work, a multi-pedestrian detector is learned
with a mixture of deformable part-basedmodels
to effectively capture the unique visual patterns appearing in multiple nearby pedestrians. 
The training data is labeled as usual, i.e. a bounding box for each pedestrian. 
The spatial configuration patterns of multiple nearby pedestrians are learned and clustered into mixture component.  
In the multi-pedestrian detector, each single pedestrian is specifically designed as a part, 
called pedestrian-part. 
A new probabilistic framework is proposed to model the configuration relationship between results of multi-pedestrian detection and 1-pedestrian detection.
With this framework, multi-pedestrian detection results are used to refine 1-pedestrian detection results.
\section{A github homepage}
We set up a github homepage \url{https://github.com/WilsonWangTHU/pedestrian-detection-thu2015}.\\
\indent And currently we are revising the code for the first stable commit.
\section{What next?}
There are several few more points possible to be revised.
\subsection{A Convolutional implementation?}
Clearly the pedestrian detection could do more, like pipline a Fast-RCNN to further detect the pose and clothes of the pedestrians.
It is clear the results of \cite{W5} could be further used as a bounding box or ROI.
I have experience of using Fast-RCNN and caffe pipeline, 
and clearly the results of the \cite{W5} could be used as the input of a new deep network.
\subsection{Contextual information}
Convolutional network could further improved if we could use some additional contextual information.
The work of \cite{W5} did not, and it could possible be helpful.
\bibliographystyle{IEEEtran}
\bibliography{Ref}
\end{document}
